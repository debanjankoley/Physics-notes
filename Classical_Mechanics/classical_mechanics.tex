\documentclass{article}
\usepackage[T1]{fontenc}
\usepackage{bm}
\newcommand{\uvec}[1]{\bm{\hat{#1}}}

\title{Classical Mechanics}
\author{Debanjan Koley}
\date{July 2024}
\begin{document}
\maketitle
\section*{Important results}
\section{Common 2D coordinate systems}

\subsection{Cartesian coordinate system}
\begin{itemize}
    \item Position, $\bm{\vec{r}} = x\uvec{i} + y\uvec{j}$
    \item Velocity, $\bm{\vec{v}} = \dot{x}\uvec{i} + \dot{y}\uvec{j}$
    \item Accelaration, $\bm{\vec{a}} = \ddot{x}\uvec{i} + \ddot{y}\uvec{j}$
    \item Kinetic energy = $\frac{1}{2}m(\bm{\vec{v}\cdot\vec{v}}) = \frac{1}{2}m(\dot{x}^2 + \dot{y}^2)$
\end{itemize}

\subsection{Polar coordinate system}
\begin{itemize}
    \item Position, $\bm{\vec{r}} = r\uvec{r}$
    
    where, $\uvec{r} = cos\theta\uvec{i} + sin\theta\uvec{j}$ ,
    \item Velocity, $\bm{\vec{v}} = \dot{r}\uvec{r} + r\dot{\theta}\uvec{\theta}$
    
    where, $\uvec{\theta} = -sin\theta\uvec{i} + cos\theta\uvec{j}$
    \item Accelaration, $\bm{\vec{a}} = (\ddot{r} - r\dot{\theta}^2)\uvec{r} + (r\ddot{\theta} + 2\dot{r}\dot{\theta})\uvec{\theta}$
    \item Kinetic energy = $\frac{1}{2}m(\bm{\vec{v}\cdot\vec{v}}) = \frac{1}{2}m(\dot{r}^2 + r^2\dot{\theta}^2)$
\end{itemize}

\newpage
\section*{Constrained motion}
Constraints are restraints imposed on the motion or location, 
or both of a system of particles.Constrained motion occurs when 
an object is forced to move in a specific manner.

Constraint can be divided in two categories:
\begin{description}
    \item[Holonomic constraint:] Holonomic constraints can be expressed as an equation that involves only the spatial coordinates $q_i$
    of the system and the time t.   
    \item[Non-holonomic constraint:] Non-holonomic constraints cannot be written as an equation between coordinates.
\end{description}

\section*{Generalized coordinates}
Generalized coordinates are a set of parameters used to represent the configuration of a system. Instead of using a particular 
set of coordinates, we use generalized coordinates $(q_i)$ which may be cartesian, polar, angles, or various combinations of them.


\end{document}