\documentclass{article}
\usepackage[T1]{fontenc}
\usepackage{bm}
\newcommand{\uvec}[1]{\bm{\hat{#1}}}

\title{Classical Mechanics}
\author{Debanjan Koley}
\date{July 2024}
\begin{document}
\maketitle
\section*{Important results}
\section{Common 2D coordinate systems}

\subsection{Cartesian coordinate system}
\begin{itemize}
    \item Position, $\bm{\vec{r}} = x\uvec{i} + y\uvec{j}$
    \item Velocity, $\bm{\vec{v}} = \dot{x}\uvec{i} + \dot{y}\uvec{j}$
    \item Accelaration, $\bm{\vec{a}} = \ddot{x}\uvec{i} + \ddot{y}\uvec{j}$
    \item Kinetic energy = $\frac{1}{2}m(\bm{\vec{v}\cdot\vec{v}}) = \frac{1}{2}m(\dot{x}^2 + \dot{y}^2)$
\end{itemize}

\subsection{Polar coordinate system}
\begin{itemize}
    \item Position, $\bm{\vec{r}} = r\uvec{r}$
    
    where, $\uvec{r} = cos\theta\uvec{i} + sin\theta\uvec{j}$
    \item Velocity, $\bm{\vec{v}} = \dot{r}\uvec{r} + r\dot{\theta}\uvec{\theta}$
    
    where, $\uvec{\theta} = -sin\theta\uvec{i} + cos\theta\uvec{j}$
    \item Accelaration, $\bm{\vec{a}} = (\ddot{r} - r\dot{\theta}^2)\uvec{r} + (r\ddot{\theta} + 2\dot{r}\dot{\theta})\uvec{\theta}$
    \item Kinetic energy = $\frac{1}{2}m(\bm{\vec{v}\cdot\vec{v}}) = \frac{1}{2}m(\dot{r}^2 + r^2\dot{\theta}^2)$
\end{itemize}

\section{Common 3D coordinate systems}

\subsection{Cartesian coordinate system}
\begin{itemize}
    \item Position, $\bm{\vec{r}} = x\uvec{i} + y\uvec{j} + z\uvec{k}$
    \item Velocity, $\bm{\vec{v}} = \dot{x}\uvec{i} + \dot{y}\uvec{j} +\dot{z}\uvec{k}$
    \item Accelaration, $\bm{\vec{a}} = \ddot{x}\uvec{i} + \ddot{y}\uvec{j} + \ddot{z}\uvec{k}$
    \item Kinetic energy = $\frac{1}{2}m(\bm{\vec{v}\cdot\vec{v}}) = \frac{1}{2}m(\dot{x}^2 + \dot{y}^2 + \dot{z}^2)$
\end{itemize}

\subsection{Cylindrical coordinate system}
\begin{itemize}
    \item Cartesian coordinates in terms of Cylindrical coordinates,
    \begin{center}
        $x = rcos\phi,$ 
        
        $y = rsin\phi,$ 
        
        $z = z$
    \end{center}
    \item Transformation table for unit vectors:
    \begin{center}
    \begin{tabular}{ c | c | c | c}
                   & $\uvec{r}$ & $\uvec{\phi}$ & $\uvec{z}$\\
        \hline
        $\uvec{i}$ & $cos\phi $ & $-sin\phi$ & 0\\
        \hline  
        $\uvec{j}$ & $sin\phi $ & $cos\phi$ & 0\\
        \hline
        $\uvec{k}$ & 0 & 0 & 1\\
    \end{tabular}
    \end{center}
    \item Position, $\bm{\vec{r}} = r\uvec{r} + z\uvec{z}$
    \item Velocity, $\bm{\vec{v}} = \dot{r}\uvec{r} + r\dot{\phi}\uvec{\phi} + \dot{z}\uvec{z}$
    \item Accelaration, $\bm{\vec{a}} = (\ddot{r} - r\dot{\phi}^2)\uvec{r} + (r\ddot{\phi} + 2\dot{r}\dot{\phi})\uvec{\phi} + \ddot{z}\uvec{z}$
    \item Kinetic energy = $\frac{1}{2}m(\bm{\vec{v}\cdot\vec{v}}) = \frac{1}{2}m(\dot{r}^2 + r^2\dot{\phi}^2 + \dot{z}^2)$
\end{itemize}

\subsection{Spherical coordinate system}
\begin{itemize}
    \item Cartesian coordinates in terms of Spherical coordinates,
    \begin{center}
        $x = rsin\theta cos\phi,$ 
        
        $y = rsin\theta sin\phi,$ 
        
        $z = rcos\theta$
    \end{center}
    \item Transformation table for unit vectors:
    \begin{center}
    \begin{tabular}{ c | c | c | c}
                   & $\uvec{r}$ & $\uvec{\theta}$ & $\uvec{\phi}$\\
        \hline
        $\uvec{i}$ & $sin\theta cos\phi $ & $cos\theta cos\phi$ & $-sin\phi$\\
        \hline  
        $\uvec{j}$ & $sin\theta sin\phi $ & $cos\theta sin\phi$ & $cos\phi$\\
        \hline
        $\uvec{k}$ & $cos\theta$ & $-sin\theta$ & 0\\
    \end{tabular}
    \end{center}
    \item Position, $\bm{\vec{r}} = r\uvec{r}$
    \item Velocity, $\bm{\vec{v}} = \dot{r}\uvec{r} + r\dot{\theta}\uvec{\theta} + rsin\theta\dot{\phi}\uvec{\phi}$
    \item Accelaration, $\bm{\vec{a}} = (\ddot{r} - r\dot{\theta}^2 - rsin^2\theta\dot{\phi}^2)\uvec{r} + (r\ddot{\theta} + 2\dot{r}\dot{\theta} - rsin\theta cos\theta\dot{\phi}^2)\uvec{\theta} + (\dot{r}sin\theta\dot{\phi} + 2rcos\theta\dot{\theta}\dot{\phi} + rsin\theta\ddot{\phi})\uvec{\phi}$
    \item Kinetic energy = $\frac{1}{2}m(\bm{\vec{v}\cdot\vec{v}}) = \frac{1}{2}m(\dot{r}^2 + r^2\dot{\theta}^2 + r^2sin^2\theta\dot{\phi}^2)$
\end{itemize}

\newpage
\section*{Constrained motion}
Constraints are restraints imposed on the motion or location, 
or both of a system of particles.Constrained motion occurs when 
an object is forced to move in a specific manner.

Constraint can be divided in two categories:
\begin{description}
    \item[Holonomic constraint:] Holonomic constraints can be expressed as an equation that involves only the spatial coordinates $q_i$
    of the system and the time t.   
    \item[Non-holonomic constraint:] Non-holonomic constraints cannot be written as an equation between coordinates.
\end{description}

\section*{Generalized coordinates}
Generalized coordinates are a set of parameters used to represent the configuration of a system. Instead of using a particular 
set of coordinates, we use generalized coordinates $(q_i)$ which may be cartesian, polar, angles, or various combinations of them.

\newpage
\section*{Calculus of variation}
In calculus of variation we find a function or a curve, for which the given functional has a stationary value.
\subsection*{Euler equation}
 
$$S = \int_{x_1}^{x_2}f(y,\dot{y},x)dx $$
\end{document}